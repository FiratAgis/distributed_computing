%\documentclass[10pt,notes]{beamer}       % print frame + notes
%\documentclass[10pt, notes=only]{beamer}   % only notes
\documentclass[11pt]{beamer}              % only frames

%%%%%% IF YOU WOULD LIKE TO CREATE LECTURE NOTES COMMENT OUT THE FOlLOWING TWO LINES
%\usepackage{pgfpages}
%\setbeameroption{show notes on second screen=bottom} % Both

\usepackage{graphicx}
\DeclareGraphicsExtensions{.pdf,.png,.jpg}
\usepackage{color}
\usetheme{winslab}
\usepackage[utf8]{inputenc}
\usepackage[english]{babel}
\usepackage{amsmath}
\usepackage{amsfonts}
\usepackage{amssymb}




\usepackage{algorithm2e,algorithmicx,algpseudocode}
\algnewcommand\Input{\item[\textbf{Input:}]}%
\algnewcommand\Output{\item[\textbf{Output:}]}%
\newcommand\tab[1][1cm]{\hspace*{#1}}

\algnewcommand{\Implement}[2]{\item[\textbf{Implements:}] #1 \textbf{Instance}: #2}%
\algnewcommand{\Use}[2]{\item[\textbf{Uses:}] #1 \textbf{Instance}: #2}%
\algnewcommand{\Trigger}[1]{\Statex{\textbf{Trigger:} (#1)}}%
\algnewcommand{\Events}[1]{\item[\textbf{Events:}] #1}%
\algnewcommand{\Need}[1]{\item[\textbf{Needs:}] #1}%
\algnewcommand{\Event}[2]{\Statex \item[\textbf{On#1:}](#2) \textbf{do}}%
\algnewcommand{\Trig}[3]{\State \textbf{Trigger}  #1.#2 (#3) }%
\def\true{\textbf{T}}
\def\false{\textbf{F}}


\author[Fırat Ağış]{Fırat Ağış\\\href{mailto:firat@ceng.metu.edu.tr}{firat@ceng.metu.edu.tr}}
%\author[J.\,Doe \& J.\,Doe]
%{%
%  \texorpdfstring{
%    \begin{columns}%[onlytextwidth]
%      \column{.45\linewidth}
%      \centering
%      John Doe\\
%      \href{mailto:john@example.com}{john@example.com}
%      \column{.45\linewidth}
%      \centering
%      Jane Doe\\
%      \href{mailto:jane.doe@example.com}{jane.doe@example.com}
%    \end{columns}
%  }
%  {John Doe \& Jane Doe}
%}

\title[WINS Beamer Template]{Implementing Mutual Exclusion in Shared Memory}
\subtitle[Short SubTitle]{Peterson's and Bakery Algorithm}
%\date{} 

\begin{document}

\begin{frame}[plain]
\titlepage
\note{In this talk, I will present .... Please answer the following questions:
\begin{enumerate}
\item Why are you giving presentation?
\item What is your desired outcome?
\item What does the audience already know  about your topic?
\item What are their interests?
\item What are key points?
\end{enumerate}
}
\end{frame}

\begin{frame}[label=toc]
    \frametitle{Outline of the Presentation}
    \tableofcontents[subsubsectionstyle=hide]
\note{ The possible outline of a talk can be as follows.
\begin{enumerate}
\item Outline 
\item Problem and background
\item Design and methods
\item Major findings
\item Conclusion and recommendations 
\end{enumerate} Please select meaningful section headings that represent the content rather than generic terms such as ``the problem''. Employ top-down structure: from general to more specific.
}
\end{frame}
%
%\part{This the First Part of the Presentation}
%\begin{frame}
%        \partpage
%\end{frame}
%
\section{The Problem}
%\begin{frame}
%        \sectionpage
%\end{frame}

\begin{frame}{The problem}
\begin{block}{Race Conditions} 
Race conditions occur in concurrent systems when the outcome of a process depends on the timing or sequence of other events. This can lead to unpredictable behavior, data corruption, or system failures. Detecting and preventing race conditions is essential for ensuring the correctness and reliability of concurrent software systems.
\end{block}
\note{}
\end{frame}

\section{The Contribution}
\begin{frame}
\frametitle{Mutual Exclusion}
\framesubtitle{}
Mutual exclusion is the most common solution for race conditions in terms of mutual memory. 
\vspace{0.2in}
\begin{itemize}
\item SharedExclusion
\begin{itemize}
	\item Peterson's Algorithm
	\item Bakery Algorithm
\end{itemize}
\end{itemize}

\end{frame}


\section{Motivation/Importance}
\begin{frame}
\frametitle{Motivation/Importance}
Disturbed systems with many process, while working on the same data, it is common for many processes to want to access or modify it at the same time. But a process changing the data while the other is reading it might create unwanted effects. Mutual exclusion is a common way of handling these effects. This enables multiple processes to work on the same piece of data, which is one of the most fundamental capabilities requested from a distributed system.
\end{frame}

\section{Definitions/Background}

\frame{
\frametitle{Definitions}

\begin{itemize}
\item When a process wants to modify a piece of data that is being accessed by others, it creates a \textbf{race condition}. 
\item Regions of code that modifies the shared memory, meaning regions that might cause race conditions are called the \textbf{critical section}.
\item \textbf{Mutual exclusion} is the methodology of allowing only a single process to enter a critical section at a time.
\end{itemize}
}

\begin{frame}{Background}
My comparisons are limited to the contemporaries of my algorithms.
\begin{itemize}
	\item Dijkstra (1965) \cite{Dijkstra_1965}
	\item Knuth (1966) \cite{Knuth_1966}
	\item de Bruijn (1967) \cite{deBruijn_1967}
	\item Dijkstra (1968) \cite{Dijkstra_1968}
	\item Eisenberg and McGuire (1972) \cite{Eisenberg_McGuire_1972}
\end{itemize}
\end{frame}


\section{Contribution}
\subsection{Peterson's Algorithm}
\begin{frame}{SharedExclusion Algorithm 1}
\framesubtitle{Peterson's Algorithm}
Peterson's Algorithm \cite{Peterson_1981} prevents 2 processes from entering the critical section in the same time. It achieves this by a way of each process giving way to other. When the process comes to the critical section, it first allows the other process to continue to the critical section and only moves forward if the other is not currently trying to enter the critical section or if the other process finishes its work within the critical section. While it is a gentleman's way of handling shared memory exclusion, its main limitation lies in the fact that it only works for 2 processes.
\end{frame}

\begin{frame}{SharedExclusion Algorithm 1}
\framesubtitle{Peterson's Algorithm}
\begin{algorithm}[H]
	\scriptsize
	\def\algorithmlabel{Peterson's}
    \caption{\algorithmlabel\ algorithm}
    \label{alg:peterson}
    \KwData{Processes $p_i$ where $i \in \lbrace 0, 1 \rbrace$}
    initialization\;
    $\texttt{turn} \leftarrow \false $\;
    $\texttt{waiting}[2]$\;
    \BlankLine
    algorithm\;
    \If{$p_i$ wants to enter \textbf{CS}}{
    	$\texttt{waiting}[i] \leftarrow \true$\;
    	$\texttt{turn} \leftarrow 1 - i$\;
    	\BlankLine
    	\lWhile{$waiting[1 - i] = \true$ and $turn = 1-i$}{no-op}
    	\BlankLine
    	$p_i$ enters \textbf{CS};
    	\BlankLine
    	$p_i$ exits \textbf{CS};
    	\BlankLine
    	$\texttt{waiting}[i] \leftarrow \false$\;
    }

\end{algorithm}
\end{frame}


\subsection{Bakery Algorithm}

\begin{frame}
\frametitle{SharedExclusion Algorithm 2}
\framesubtitle{Bakery Algorithm}
Bakery Algorithm \cite{Lamport_1974} simulates the activity of waiting in a line in its namesake to get your order. Every process takes a ticket with a number on it. When their number flashes on the screen, they come and give their order. If two processes have the same number on their ticket, the conflict is resolved through seniority, just like letting an elderly person go before you when both of you arrive at the same time.

\end{frame}

\begin{frame}{SharedExclusion Algorithm 2}
\framesubtitle{Bakery Algorithm}
\begin{algorithm}[H]
	\scriptsize
	\def\algorithmlabel{Bakery}
    \caption{\algorithmlabel\ algorithm}
    \label{alg:bakery}
    \KwData{Processes $p_i$ where $i \in [1,n]$}
    initialization\;
    $\texttt{entering}[n]$\;
    $\texttt{ticket}[n]$\;
    \BlankLine
    algorithm\;
    \If{$p_i$ wants to enter \textbf{CS}}{
    	$\texttt{entering}[i] \leftarrow \true$\;
    	$\texttt{ticket}[i] \leftarrow \max (\texttt{ticket}) + 1$\;
    	$\texttt{entering}[i] \leftarrow \false$\;
    	\BlankLine
    	\For{$j \leftarrow 1$ \KwTo $n$}{
    		\lWhile{$entering[j] = \true $}{no-op}
    		\lWhile{$ticket[j] \neq 0$ and $(ticket[j],j)<(ticket[i],i)$}{no-op}
    	}
    	\BlankLine
    	$p_i$ enters \textbf{CS};
    	\BlankLine
    	$p_i$ exits \textbf{CS};
    	\BlankLine
    	$\texttt{ticket}[i] \leftarrow 0$\;
    }

\end{algorithm}
\end{frame}

\subsection{Implementation Details}
\begin{frame}
\frametitle{Implementation Details}
\framesubtitle{ }
\begin{itemize}
	\item Central vs Distributed Implementation.
	\item Non-event driven initialization.
	\item Leader election.
\end{itemize}
\end{frame}


\section{Results}

\subsection{Testing}
\begin{frame}
\frametitle{Testing}
\framesubtitle{}
\begin{itemize}
	\item Peterson's Algorithm
		\begin{itemize}
			 \item Topology with two nodes.
		\end{itemize}
	\item Bakery Algorithm
		\begin{itemize}
			\item Well-connected topologies ($O(n^2)$ connections)
			\item Topologies where only connections are between the leader and others.
			\item Two directional ring topologies without shortcuts.
			\item One-directional ring topologies without shortcuts.
		\end{itemize}
\end{itemize}
\end{frame}

\subsection{Future Work}
\begin{frame}
\frametitle{Future Work}
\framesubtitle{}

\begin{itemize}
	\item Increase the event drivenness of the current implementation.
	\item Implement distributed versions of the algorithms and compare them to current implementations.
\end{itemize}

\end{frame}

\section{Conclusions}
\begin{frame}
\frametitle{Conclusions}
\framesubtitle{What was achieved?}
\begin{itemize}
	\item We learned about mutual exclusion.
	\item Functional (but not at full potential) algorithms were added to the AHCv2 platform.
	\item I gained experience in event-driven and distributed computing.
\end{itemize}
	
\end{frame}

\section*{References}
\begin{frame}{References}
\tiny
\bibliographystyle{IEEEtran}
\bibliography{refs}
\end{frame}

\thankslide

\end{document}